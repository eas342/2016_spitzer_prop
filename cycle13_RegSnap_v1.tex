%% cycle13_RegSnap.tex
%
%  Spitzer Space Telescope Cycle-13 Proposal Template
%
%  Use this template for Cycle-13 Regular GO and Snapshot proposals.
%  No style file is required.
%
%  Version 1.0    9 Feb 2016
%
%%%  For Spitzer proposal preparation resources please visit 
%    the proposal kit web page: 
%
%%%  http://ssc.spitzer.caltech.edu/warmmission/propkit/ 
%
%  In particular, please read the Cycle-13 Call for Proposals (CP).
%  It is the definitive document that describes the requirements
%  necessary for your proposal. 
%
%    Please address all questions regarding the proposal call
%    and observation planning to the Helpdesk at 
%
%%%  help@spitzer.caltech.edu
%
%
%
%%%%%%%%%%%%%%%%%%%%%%%%%%%%%%%%%%%%%%%%%%%%%%%%%%%%%%%%%%%%%%%%%%%%%%%%
%
%  The template begins here.  The font must be 12 point and the margins
%  must be at least 1-inch on all sides. 
%  Don't override this.  
%
% If you compile this and find that the text is "mushed up against
% the top of the page", the default paper size for your installation 
% of latex is A4.  In order to override this, do:
% > latex texfile # where the manuscript is in a file named texfile.tex
% > dvips -Ppdf -t letter -o texfile.ps texfile
% finally, to get nice (non-blurry, searchable) pdf do:
% > ps2pdf14 texfile.ps  texfile.pdf
% if you do not have ps2pdf14, please ask your sysadmin to install it.


\documentclass[letterpaper,12pt]{article}
\usepackage{epsfig}
\textwidth=6.5in
\textheight=9.5in
\topmargin=-0.75in
\oddsidemargin=0.0in
\evensidemargin=0.0in

\pagestyle{myheadings}
\usepackage{natbib}
\include{shortcuts}

% Please update the following line with the title of your proposal
% and your Author name (with "et al." if more than two authors).

\markright{Disintegrating Planet Studies, E.\ Schlawin et al.}
\pagenumbering{arabic}

\begin{document}

%\noindent {\bf Sections and Page limits:\\
%Science Plan\\
%Science Justification:  Small: 1 page; Medium: 2 pages;  Large: 3 pages; Snap $\ge$500 hr: 3 pages\\
%Technical Justification:  Small: 1 page; Medium: 1 page;  Large: 2 pages; Snap $\ge$500 hr: 2 pages\\
%Figures, Tables, References:  Small: 1 page; Medium: 2 pages;  Large: 3 pages; Snap $\ge$500 hr: 3 pages\\
%Additional Required Sections \\
%Summary of Existing Programs:  1 page\\
%Observation Summary Table:  no page limit\\
%Modification of the Proprietary Period:  no page limit\\
%Summary of Duplicate Observations:  no page limit\\
%Summary of Scheduling Constraints/ToOs:  no page limit\\}
% We'll need about 33 hr, so It's a "Medium Proposal"

\section{Science Plan}

%The Science Plan includes three parts:\newline
%
%\noindent SCIENCE JUSTIFICATION\\
%TECHNICAL JUSTIFICATION\\
%FIGURES, TABLES \& REFERENCES\\
%
%The total page limit for the Science Plan is 3 pages for small GO or Snapshot proposals 
%($<= 10$ hours), 5 pages for medium GO or Snapshot proposals (10 - 100 hours),
%8 pages for large GO or Snapshot proposals (100 - 500 hours) and 8 pages for Snapshot
%proposals $\ge$ 500 hours. Detailed page limits and descriptions of required 
%sections are provided in Chapter 5 of the Call for Proposals.\newline
%
%For joint observatory proposals one additional page per 
%joint observatory should be included to describe the 
%technical justification of the joint observations. Therefore 
%the total number of allowed pages increases by one page for 
%each joint observatory included.  

\subsection{Science Justification}

\noindent {\bf Disintegrating Bodies - A New Class of Planets}\\

The Kepler Observatory has ushered in a new era of planet discovery and characterization, and dramatically increased the inventory of known planets and planet candidates.
One of the fascinating new classes of planets uncovered by the Kepler observatory is that of disintegrating rocky bodies,
including KIC 12557548b \citep{rappaport}, KOI 2700b \citep{rappaport2014KOI2700}, K2-22b \citep{sanchis-ojedak2-22}, WD 1145+017 \citep{vanderburg2015wdDisintegrating} and KIC 8462852 \citep{boyajian846}.
The transiting events for WD 1145+017 and KIC 8462852 are wildly stochastic and unpredictable, so they make followup with Spitzer difficult.
KIC 12557548b (hereafter shortened to KIC 1255b) and K2-22b, on the other hand, are periodic and have well-measured ephemerides.
We therefore focus on these two disintegrating systems KIC 1255b  and K2-22b for Spitzer IRAC photometry.
\textbf{Disintegrating systems offer a rare opportunity to study planets that have been peeled away layer by layer}, and complement studies of white dwarfs, which are sensitive to the bulk compositions of accreted planetesimals \cite[e.g.][]{jura2003wdPollution}.\newline

\noindent {\bf Particle Size and Composition}\\

The K2-22b and KIC 1255b systems show broadband flux decreases that likely comes from the scattering of dust particles -- gas absorption from atomic or ionized gas would not have enough opacity to create the large broadband transits observed by Kepler photometry \citep[0.42 $\mu$m to 0.90 $\mu$m bandpass;][]{koch2010keplerChar}.
The creation and/or destruction of dust particles cause variable transit depths, shown in Figure \ref{fig:exKeplerCurves}.
The transit depths can vary from $\sim$0\% (undetected) to 1.3\% depending on the level of disintegration activity.
The disintegration is correlated to stellar magnetic activity for the KIC 1255 system \citep{kawahara2013starspots}, but this effect may be lessened by occultations of starspots by planetary debris \citep{croll2015starspots}.

The particle sizes and composition of the debris escaping these disintegrating bodies are still largely unknown, but there are some indications that particle sizes change over time.
\citet{bochinski2015evolving} find that the spectral slope of KIC 1255b spectrum changes between two nights and that the spectral slope between the $u'$, $g'$ and $z'$ bands was flatter during a shallow transit than a deeper transit. 
\cite{sanchis-ojedak2-22} find that the spectral slope of K2-22b also shows a steeper spectral slope during a deeper $\sim 0.8\%$ event whereas a flat spectrum during a shallow $\sim$ 0.4\% event.
The slope of the transmission spectra for these disintegrating bodies is sensitive to particle size with a steep spectral slope indicating small particle sizes and a shallow spectral slope indicating large particle sizes.
We aim to measure the particle sizes of the debris using 4.5$\mu$m photometry to characterize the particle sizes of the debris, to higher precision than has been achieved with ground-based optical data.
So far, there is only a lower limit on the particle sizes during medium and shallow transits.
A lower limit of $\sim$0.5$\mu$m was placed on the dust particle sizes for both medium depth depth transits of KIC 1255b using simultaneous $K$ band (2.4$\mu$m) and Kepler broadband (0.42 to 0.90$\mu$m ) photometry \citep{croll2014}.
Similarly, \citep{schlawin2016kic1255} find a lower limit of 0.5$\mu$m sized grains for olivine and pyroxene compositions using low resolution near-infrared spectroscopy (0.9 to 2.4$\mu$m) and optical photometry \citep{schlawin2016kic1255}.
\textbf{Spitzer IRAC photometry is needed to study the large ($0.5\mu$m) dust particle escaping from KIC 1255b and K2-22b.}\newline



%\noindent {\bf Underlying Planet}\\
%
%Little is known about the underlying planets from which the disintegration occurs.
%For KIC 1255b, there are upper limits on the mass of 1.2 R$_{Jup}$ from radial velocity observations \citep{croll2014} and a radius of ~4600 km for an albedo of 1 \citep{vanWerkhoven2014} from the lack of secondary eclipse.
%Similarly, for K2-22b, the upper limit on the mass is 1.2 R$_{Jup}$ \citep{sanchis-ojedak2-22} from radial velocity data but this does not constrain the properties of the planet other than ruling out stellar-mass objects as causing the transit-like features.
%In the event of weak disintegration activity and small particle sizes, Spitzer photometry can put new constraints on the size of the planet \newline

\noindent {\bf Tail Geometries}\\

The K2-22 and KIC 1255 systems are interesting to compare and contrast because they display many of the same stochastic transit depth behaviors, but have different transit profiles.


%The panels have broad scientific expertise.  Include background 
%on the subject you are studying, in particular to help investigators 
%not in your sub-field understand the importance of the research. \newline

%\noindent {\bf General Advice}\\
%If you would like a step-by-step walkthrough of how to submit a
%proposal using Spot, and hints and tips for doing so, please see
%the Observation Planning Cookbook, available on the SSC website.\newline
%
%Please read the Call for Proposals (CP). It is
%the definitive document that describes the requirements
%necessary for your proposal. \newline 
%
%You must submit a PDF version of your proposal. \newline
%
%\noindent {\bf Note that there is no abstract or title or list of people
%in the Science Plan.}  The abstract and title and PI/Co-Is are part of
%the coversheet, which is all information you enter into Spot
%when you submit your proposal.  This coversheet is generated by
%the SSC for the final submission; if you want your very own
%copy, see under the ``File'' menu in the Proposal Tool in Spot.\newline
%
%Do not reduce the size of the margins or fonts.  This annoys the reviewers.
%Unusual fonts do not render on all systems.  Don't change the
%font size of the major headers --  large section header
%fonts make the proposal easier to scan. \newline
%
%Every section listed here has a purpose so please include all of them.
%Do not attach any additional material, e.g. letters of endorsement.

\clearpage

\subsection{Technical Justification}

\begin{table}[htbp]
   \centering
   %\topcaption{Table captions are better up top} % requires the topcapt package
   \begin{tabular}{@{} lcr @{}} % Column formatting, @{} suppresses leading/trailing space
      \multicolumn{2}{c}{} \\
      N   & KIC 1255b & K2-22b \\
      \hline
      1      & 90.6 & 94.6 \\
   \end{tabular}
   \caption{\textbf{Insert Same Table 1 from the HST Proposal.}}
   \label{tab:probabilities}
\end{table}

\noindent {\bf Time Critical Requirements}\\

Our observations are \textbf{time critical} and must occur during transit so we refer to the ephemerides in Observation Summary.
Furthermore, there have been quiescent periods observed for KIC 1255b of about ~30 days in duration \citep{vanWerkhoven2014}.
We request that the transit observations are separated by 30 days to ensure that they don't all fall within quiescent period where all transit depths are shallow.
\newline

\noindent {\bf Signal To Noise}\\

This section contains the technical details for your program. 
The Technical Justification should contain details of your planned observations, 
descriptions of scheduling constraints, data analysis plans and a 
description of how the technical details were validated. For proposals that 
include generic targets or targets of opportunity, this section should also 
include a discussion of the provenance and availability of the 
proposed targets.\newline

Note that you do NOT need to cut-and-paste the AORs into your
proposal; you submit your proposal using Spot, and the AORs in
the AOR window when you submit are the ones attached to your
proposal.  If you accidentally submit the wrong ones, don't worry;
you can update your proposal as often as you want before
the deadline.\newline

Make your plans clear by summarizing information in tables
and using the Observation Summary Table (section 3) to your advantage.\newline

Include brightness estimates 
for your targets; for example, see the Observation Summary Table
in section 3.  Based on those brightness estimates, include
your estimate and justification of what kind of S/N values you
need to accomplish your science.  Finally, include your
assessment of background levels (based on Spot, ISSA plates,
etc.) and explain (if necessary) how you can see your targets
despite the background. You should include numbers from the 
online SENS-PET (using the appropriate background setting) 
that justify going as deeply (or as shallowly) as you have.\newline

You should explain why you don't need to worry
about confusion limits or saturation limits, as appropriate, or
(alternatively) why you think you can get around these limits.\newline

Explain the observation strategy you decided upon for your
program, as well as the reasons for it.  For example, ``We
created several small IRAC tiles to cover the region of
interest and constrained them loosely with a `group within'
constraint in order to cover the same area regardless of
rotation angle.''  The proposal should include something about
the AOR status, e.g., ``We are submitting final AORs for this
program,''  or, based on instructions in the CP, perhaps
instead, ``Because this is a large proposal, we are only
submitting representative AORs for this program and will need to
submit final AORs after the program is accepted.'' \newline

If relevant, include a discussion of the bright objects that
your observation covers (as defined via Spot's bright object
overlay), and include specific justification as to why you
want to do this observation anyway.\newline

Consult the ``Best Observing
Practices'' sections in the Spitzer Observer's Manual $-$ Warm Mission
(Warm SOM). You should include an explanation of anything that you 
want to do that is contrary to these ``Best Observing Practices.''  For
example, the Best Observing Practices recommends dithers rather
than repeats, so you might write, ``We are using 
repeats rather than dithers because we are doing time series
monitoring and wish to place our object on the same part of the
array.''\newline

Justify any use of low-impact Target of Opportunity observations. \newline

If you have many targets, you may wish to include a separate 
table rather than a text summary here. Organize the
table in whatever way makes the most sense to you.   
Please note: this table is NOT the Observation
Summary Table described in the CP and below.  \newline

You can also put information about backgrounds or expected target 
fluxes (or both) into the Observation Summary Table (section
3), because the Observation Summary Table does not have page
limits.  \newline

Be sure to include here a detailed assessment of duplications and 
constraints, just {\em summaries} of which are placed in
sections 5 and 6.\newline

Finally, have a detailed description of how you will handle the
data you get, and who will do what from the list of Co-Is.  You
might want a description of who will lead the entire effort, and
any additional management discussion for large collaborations. 

\clearpage
\subsection{Figures, Tables \& References}

\begin{figure}
\centering
\includegraphics[width=0.9\textwidth]{kepler_lightc_variable.png}
\caption{Example lightcurves for KIC 1255b (left) and K2-22b (right) from the Kepler Observatory. The transits are periodic, but highly variable depth, suggesting that the planets are disintegrating into tails of dusty effluents.}\label{fig:exKeplerCurves}
\end{figure}

%Figures are an excellent way to convey information to your reviewers.
%Captions, tables and references may be in 10-point font.  Figures and tables 
%can be embedded within the narrative text in the Science Plan or 
%segregated into a separate section. \newline

\bibliographystyle{aasjournal}
\bibliography{spitz13prop}

%\begin{figure}[ht]
% \begin{center}
% \epsfig{file=file1.eps, width=10cm}
%\end{center}
%\caption{Sample caption for one method of including figures.}
%\end{figure}

%\begin{figure*}
%    \centering
%    \includegraphics[width=7.5cm, angle=0]{file1.ps}
%    \includegraphics[width=7.5cm, angle=0]{file2.ps}
%\caption{Sample caption for another completely different but equivalent method of including figures.}
%\end{figure*}


\noindent {\bf In addition to the Science Plan, the following 1-page section is required:}

\section{Summary of Existing Programs}

%No more than one page should be used to summarize your current
%involvement as a Principal Investigator or Technical Contact on existing 
%Spitzer Space Telescope research programs. This applies to the PI and 
%key Co-Is on the proposal. The proposer should indicate the status of 
%each Spitzer GTO, GO, Legacy, DDT, Archival or Theoretical program and 
%any publications resulting from the program(s). For observing programs, 
%include the status of the data analysis effort.
%
%Proposers that are the PI/Technical contact for multiple Spitzer programs 
%are not required to provide a detailed status for every program. They 
%should provide a summary that includes the number of programs, overall 
%status (e.g. 75\% observed, 50\% data analysis complete, 20 papers published, 
%20 papers submitted, etc.) that will allow the reviewers to understand 
%the state of the programs.\newline

\textbf{Anyone on an existing program?}
\newline
%PI J.\ Smith is also PI of GO-12 program xxxx, and is the TC for
%DDT program yyy.  The data for these programs have not yet been
%obtained.
%
%Co-I Q.\ Jones is the TC for GTO program zzz.  These data are
%discussed in 2013, ApJ, xxx, xxx.
%
%Co-I X.\ Kim is the PI of DDT program xxx.  These data have been
%processed and are anticipated to be submitted for publication
%this summer.\newline
\clearpage
\noindent {\bf The following sections are required but do NOT have page limits:}

\section{Observation Summary Table}\label{sec:ObsSumm}

%This section has no page limit. An observation summary table is required for all 
%proposals.  See the CP for the details of what is required. This section should not
%have any figures. The table can be tailored to your proposed observations
%but should at minimum, list each proposed observation with the position,
%integration time per array, map size (if larger than one field of view), 
%and estimated source fluxes. All of the targets submitted in the AORs 
%should be described in the Observation Summary Table.\newline

{\bf A Perl script that parses information from the AOR file into 
a format that can be reformatted into a table is available in the Proposal Kit.}\newline

This table should not use a microscopic font; there are no page
limits here, so if you need 10 pages to list all 400 sources that
you are observing, please don't feel that they have to be listed
in 6pt font. The table presented here is an example for the extragalactic 
First Look Survey Field.\newline

\bigskip
\begin{tabular}{lllllcc}
\hline \\ 
Target & Position & Flux     & IRAC  & Int./ & AOR & \# of \\
Field & (J2000)   & (mJy) & Band & Pixel & Duration & AORS \\
& &  & ($\mu$m) & (secs) & (hr) & \\
\hline \\ 
KIC 12557548 & 19:23:51.9+51:30:17 & 1.4 & 3.6 & 2 & 5 & 3 \\
K2-22 & 11:17:55.9+02:37:09 & 3.3 & 4.5 & 2 & 4 & 3 \\
\multicolumn{7}{c}{Each set of AORs is repeated 3 times} \\
\hline \\
\end{tabular}


There are xx hrs total requested for this observing program.\newline

%If joint time with HST or Chandra is also proposed, please summarize 
%here how many orbits or kiloseconds of time are requested.


\section{Modification of the Proprietary Period}

%This section has no page limit. The default proprietary period for General 
%Observer and Snapshot programs $< 500$ hours is 365 days. 
%Please specify any requested reduction in this proprietary period in this section.
%The default proprietary period for Snapshot programs $\ge$ 500 hours is zero days.
%A maximum of 90 days can be requested and should be justified here.
We request the standard proprietary period of 365 days.

\section{Summary of Duplicate Observations}

This section has no page limit. {\it Briefly} summarize the justification for 
any proposed duplicate observation. The details should have been provided in the
Science Plan.  Even if there are no duplicates, note this. Leopard should be 
used to find any duplications, including observations that used Instrument
Engineering Requests (IERs), since such observations will not
show up in Spot searches. More information can be found in the
Leopard User's Guide.\newline

There are two programs that include imaging of the targets in this program.
Program 10067 (PI M. Werner) and Program 11026 (PI M. Werner) both include imaging of the two targets in this program.
Neither of these previous programs is a transit/time series observation that can be used to extract information about the planet or their debris when going in front of the host stars, as discussed in the Science plan.
Therefore, this program is not a duplicate observation.

%\noindent Example: 
%
%A search using Leopard of the ROC reveals a handful of apparent
%duplications with pid xxx (AOR label ``IRAC-0000''), yyy (all
%AORs), and zzz (AOR label ``IRAC-1234'').  We believe none of
%these are true duplications; see Science Plan for further discussion.  \newline
%
%\noindent or: \newline
%
%\noindent There are no duplicate observations.


\section{Summary of Scheduling Constraints/ToOs}

This section has no page limit. Briefly summarize the justification for 
any proposed scheduling constraints. 
Also provide a summary of any ToO scheduling issues. 
The details should have been provided in the Science Plan. Even if there 
are no constraints or ToOs, note this.\newline

These observations are \textbf{time critical} and need to take place when K2-22b and KIC 1255b cross in front of their host stars .We provide an ephemeris below. For

\noindent Example:

We have placed a loose group-within constraint, spanning 10 days,
on our three AORs that cover the extended target XXX to 
minimize field rotation in the large map that will be produced by the AORs.\newline

\noindent or:\noindent 

We will be requesting 1 low-impact ToO and we have indicated this in the 
AORs.  We anticipate one new comet to be discovered this year that meets 
our criteria as discussed in the Science Plan.\newline

\noindent or:

There are no scheduling constraints or ToOs in this program.

\end{document}

